% % % % % % % % % % % % % % % % % % % % % % % % % % % % % % % % % % % % % % % % %
\section{Grundlagen}
% % % % % % % % % % % % % % % % % % % % % % % % % % % % % % % % % % % % % % % % %

In diesem Abschnitt sollen einige Grundlagen eingeführt werden, die in den folgenden Kapiteln von Bedeutung sind. Zunächst werden
einige Definitionen und Notationen bezüglich Relationen angegeben. Daraufhin wird noch einmal kurz auf Typklassen in Haskell
eingegangen.


% - - - - - - - - - - - - - - - - - - - - - - - - - - - - - - - - - - - - - - - - - - - - - - - - - - - - - - - - - - - - - - - - - - - - - - - - - - - - - -
\subsection{Relationen}
% - - - - - - - - - - - - - - - - - - - - - - - - - - - - - - - - - - - - - - - - - - - - - - - - - - - - - - - - - - - - - - - - - - - - - - - - - - - - - -

Die mathematische Grundlage und das wichtigste Werkzeug beim Erzeugen von freien Theoremen sind Relationen.

Sie werden verwendet, um Ausdrücke einander zuzuordnen, die durch Polymorphie ähnlich oder ``austauschbar'' sind.
In diesem Abschnitt werden einige Definitionen und Notationen eingeführt, die wir in späteren Abschnitten verwenden werden,
um Relationen zu beschreiben. \todo{Quellen}

Eine Relation ist auf zwei Mengen definiert, deren Elemente sie einander zuordnet. Die Schreibweise
$\mathcal{R} : T_1 \Leftrightarrow T_2$ sagt, dass $\mathcal{R}$ eine auf den Mengen $T_1$ und $T_2$ definierte
Relation ist. Sie ordnet den Elementen der ersten Menge beliebig viele Elemente der zweiten Menge zu. Man kann die Relation
auffassen als eine Menge von Paaren, wobei jedes Paar aus einem Element aus $T_1$ und einem Element aus $T_2$ besteht.

\begin{align*}
T_1 \times T_2 \subset \mathcal{R}
\end{align*}

Die Aussage $(x, y) \in \mathcal{R}$ bedeutet also, dass x und y bezüglich $\mathcal{R}$ verwandt sind. Man schreibt hierfür
auch gelegentlich $x\ R\ y$. Aus zwei Relationen kann man neue Relationen konstruieren, so liefert beispielsweise die
Verknüpfung \todo{Verknüpfung?} $;$ von zwei Relationen $\mathcal{R} : T_1 \Leftrightarrow T_2$ und
$\mathcal{S} : T_2 \Leftrightarrow T_3$ die folgende Relation.

\begin{align*}
&\mathcal{R} ; \mathcal{S} : T_1 \Leftrightarrow T_3\\
&\mathcal{R} ; \mathcal{S} := \{ (x, z) | x \in T_1, z \in T_3, \exists y : (x, y) \in \mathcal{R} \wedge (y, z) \in \mathcal{S} \}
\end{align*}


\subsubsection{Relationen und Funktionen}
%`````````````````````````````````

Wie bereits erwähnt, ordnet eine Relation einem Element ein einzelnes Element, mehrere Elemente oder gar keine Elemente zu.
Das unterscheidet sie von einer Funktion, oder besser: Das macht Funktionen zu speziellen Relationen. Eine
Funktion ist auch auf zwei Mengen definiert, doch sie ordnet jedem Element genau ein Ergebnis zu - Eingabewerte mit
mehreren Funktionswerten sind genausowenig enthalten wie Eingabewerte ohne Ergebnis.

Betrachten wir eine beliebige Funktion $f : T_1 \rightarrow T_2$, d.h. eine Funktion, die Elemente der Menge $T_1$ auf
Elemente der Menge $T_2$ abbildet. Zu jedem $x \in T_1$ ist also $f(x)$ definiert, wobei $f(x) \in T_2$. Funktionen kann
man aber auch als Relationen darstellen.

\begin{align*}
f = \{ (x, f(x)) | x \in T_1 \}
\end{align*}

Andersrum bedeutet das, dass es Relationen gibt, die man auch als Funktionen auffassen kann. Hat man beispielsweise
eine Aussage über beliebige Relationen $\forall \mathcal{R} : T_1 \Leftrightarrow T_2 . A(\mathcal{R})$, wobei A eine
Aussage ist, die von $\mathcal{R}$ abhängt, dann macht es Sinn, aus dieser Aussage eine speziellere Aussage zu folgern,
die wie folgt aussieht.

\begin{align*}
\forall f : T_1 \rightarrow T_2 . A'(f)
\end{align*}

Hierbei soll A'(f) die auf Funktionen spezialisierte Aussage sein. Aus jeder Aussage über beliebige Relationen folgt also eine
Aussage über entsprechende Funktionen. Dies ist für freie Theoreme von Bedeutung, da man sich vor allem für eine vereinfachte
Darstellung mit Funktionen interessiert.


% - - - - - - - - - - - - - - - - - - - - - - - - - - - - - - - - - - - - - - - - - - - - - - - - - - - - - - - - - - - - - - - - - - - - - - - - - - - - - -
\subsection{Haskell}
% - - - - - - - - - - - - - - - - - - - - - - - - - - - - - - - - - - - - - - - - - - - - - - - - - - - - - - - - - - - - - - - - - - - - - - - - - - - - - -

Die in dieser Ausarbeitung verwendete Bibliothek generiert freie Theoreme für in Haskell geschriebene Programme. Der Fokus
liegt dabei auf der Erweiterung um Typkonstruktorklassen, weshalb in diesem Abschnitt kurz darauf eingegangen werden soll,
wie genau eine solche Klasse in Haskell definiert wird. Ansonsten wird davon ausgegangen, dass grundsätzliche Kenntnisse der
Programmiersprache vorhanden sind, andernfalls sei an dieser Stelle auf \cite{haskell} verwiesen.

Typklassen in Haskell setzen Ad-Hoc-Polyphormismus um und können beliebig definiert werden. Eine Typklasse besteht aus
einem Namen, beliebig vielen Funktionssignaturen und gegebenenfalls aus vordefinierten Implementierungen einzelner
Funktionen. Es folgt ein einfaches Beispiel, das auch in der Prelude von Haskell enthalten ist \todo{ist es?}.

\begin{minted}{haskell}
class Show a where
   show :: a -> String
\end{minted}

Der Name dieser Beispielklasse ist \texttt{Show}. Man gibt eine Typvariable an, in diesem Fall \texttt{a}, die auch in jeder
Funktionssignatur vorkommen muss. In dieser Beispielklasse wird genau eine Funktionssignatur angegeben: Die Signatur
\texttt{a -> String} für die Funktion \texttt{show}.

Im weiteren Haskellprogramm kann nun für beliebige Datentypen eine Instanzdeklaration angegeben werden. Der folgende
Quellcode zeigt ein Beispiel für einen selbstdefinierten Datentyp.

\begin{minted}{haskell}
data Test = Test Int String

instance Show Test where
   show (Test _ s) = s
\end{minted}

Kommt nun im Programm in einem Ausdruck der Funktionsaufruf \texttt{show} vor, so wird anhand des Parametertypen
entschieden, welche Instanz dieser Klasse verwendet wird. Es kann also für jeden Typen eine komplett andere Implementierung
angegeben werden.

Nun ist es ja auch möglich, dass ein selbstdefinierter Datentyp einen Typen als Parameter erwartet, wie das folgende Beispiel
zeigt.

\begin{minted}{haskell}
data Test a = Test a String
\end{minted}

Beim Ausdruck \texttt{Test} handelt es sich in diesem Fall also nicht um einen Typen, sondern eher um einen funktionsähnlichen Ausdruck, der
einen Typen als Parameter erwartet und einen Typen zurückgibt. Man spricht hier auch von Sorten (engl. ``kinds''). Haskell
führt für Sorten eine eigene Notation ein. Alle Basistypen wie \texttt{Int}, \texttt{String}, etc. sind von der Sorte *.
Typkonstruktoren, die einen Typen als Parameter erwarten und daraus einen neuen Typen konstruieren, sind von der Sorte
$* \rightarrow *$.
Kompliziertere Sorten wie zum Beispiel $* \rightarrow (* \rightarrow *) \rightarrow *$ sind ebenfalls möglich.

Typklassen können auch für Datentypen definiert werden, die Typparameter erwarten. Das folgende Beispiel zeigt die
Definition für die Typklasse \texttt{Functor}, die für einen Typen definiert ist, der wiederum einen Typparameter erwartet.

\begin{minted}{haskell}
class Functor f where
   fmap :: (a -> b) -> f a -> f b
\end{minted}

Hier ist \texttt{f} die Typvariable. Die Signatur für \texttt{fmap} lässt erkennen, dass \texttt{f} auf einen Parameter appliziert
wird, also muss \texttt{f} von der Sorte $* \rightarrow *$ sein. Zu beachten ist hier, dass es keinen Sinn macht (und auch nicht
gestattet ist), dass die Typvariable mit unterschiedlichen Anzahlen an Argumenten auftaucht. Ebenso wäre es ungültig,
eine Instanz von \texttt{Functor} zu definieren für einen Datentyp, der keine Typparameter erwartet.
