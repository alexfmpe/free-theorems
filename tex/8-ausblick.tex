% % % % % % % % % % % % % % % % % % % % % % % % % % % % % % % % % % % % % % % % %
\section{Ausblick}
% % % % % % % % % % % % % % % % % % % % % % % % % % % % % % % % % % % % % % % % %

Es wurde bereits in Kapitel \ref{sec:striktheit-und-rekursion} darauf eingegangen, inwiefern Striktheit und der Fixpunktoperator
eine Rolle bei der Gegnerierung freier Theoreme spielen. In der Erweiterung der Bibliothek wurden diese bisher außer Acht gelassen,
sie funktioniert nur mit der \texttt{BasicLanguage} korrekt. Aber natürlich wäre es wünschenswert, auch die übrigen
Teilsprachen zu unterstützen.

%Die umgesetzte Erweiterung basierte zum großen Teil auf einer Arbeit von Voigtländer \cite{voigtlander}, deren Betrach

In der umgesetzten Implementierung ist nur die Typsorte $* \rightarrow *$ vorgesehen. Aktuelle Spracherweiterungen des
GHC Compilers ermöglichen hingegen sogar benutzerdefinierte Typsorten \cite{yorgey} \cite{atkey} \todo{richtiges zitat?}. Auch
hier wäre noch Raum für Verbesserungen.

Wirklich komfortabel wird der Umgang mit der Bibliothek erst, wenn ein angenehmes Benutzerinterface verfügbar ist.
Bei der Entstehung dieser Arbeit wurde beispielsweise ein kleines Tool geschrieben, das Zugriff auf die Funktionalitäten
der Bibliothek bietet. Es verrichtet seine Arbeit, bietet aber keinen echten Komfort beim Erzeugen der Theoreme. Zudem
läuft es lediglich in der Kommandozeile.

Angenehmer stellt sich die Verwendung dar, wenn beispielsweise ein Webinterface wie das Paket \textit{free-theorems-webui}
\cite{freetheoremswebui} verwendet wird. Nicht nur wird die Eingabe dadurch intuitiver und einfacher, es ist auch eine
übersichtlichere Darstellung des resultierenden Theorems möglich.

\textit{free-theorems-webui} importiert dabei eine Datei vordefinierter Deklarationen, sodass bekannte Typen aus der
Prelude verwendet werden können und man Funktionssignaturen zu bekannten Funktionen direkt über deren Namen
angeben kann. Bei der Umsetzung der vorliegenden Arbeit wurden Deklarationen für \texttt{Functor} und \texttt{Monad}
eingebaut. Doch nicht nur gibt es weitere wichtige Typklassen, die jetzt ebenfalls verwendet werden können, es gibt
auch viele Funktionen, die mit Typkonstruktorvariablen arbeiten und daher bisher nicht verwendbar waren.

Es bietet sich also an, die Liste der \texttt{KnownDeclarations} auszubauen und sich zu überlegen, welche
Typkonstruktorklassen und Funktionen sinnvoll sind.