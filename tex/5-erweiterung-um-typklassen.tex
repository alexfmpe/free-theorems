% % % % % % % % % % % % % % % % % % % % % % % % % % % % % % % % % % % % % % % % %
\section{Erweiterung um Typkonstruktorklassen}
% % % % % % % % % % % % % % % % % % % % % % % % % % % % % % % % % % % % % % % % %

\label{sec:erweiterung-um-typklassen}

In den bisherigen Abschnitten wurden die Grundlagen zu freien Theoremen erklärt, es wurde erläutert, wie freie Theoreme
aussehen, wenn Typkonstruktorklassen eine Rolle spielen, und es wurde ein Einblick in die Bibliothek \textit{free-theorems}
gegeben. Es sind jetzt alle Werkzeuge vorhanden, um die Bibliothek so zu erweitern, dass auch Typkonstruktorklassen möglich
sind, deren Typparameter die Sorte $* \rightarrow *$ haben. Und natürlich muss es möglich sein, Typsignaturen zu schreiben,
in denen Typvariablen auf solche Klassen beschränkt und auf Typparameter angewandt werden.

%Dieses Kapitel ist ähnlich gegliedert wie Kapitel \ref{sec:free-theorems}: Die einzelnen Abschnitte entsprechen den Funktionen,
Die Gliederung dieses Kapitels orientiert sich an Kapitel \ref{sec:free-theorems}: Die Bibliothek wird in derselben Reihenfolge
durchlaufen, und zu jeder Funktion wird erläutert, inwiefern sich Funktionsweisen ändern bzw. neue Funktionalitäten hinzugefügt
werden. Dabei wird dort, wo es sinnvoll ist, ein Einblick in die veränderten Datenstrukturen gegeben, die hierbei eine Rolle spielen.
%die bei der Verwendung der Bibliothek durchlaufen werden. Dabei wird jeweils erläutert, inwiefern sich Funktionsweisen ändern
%bzw. neue Funktionalitäten hinzugefügt werden. Insbesondere wird ein Blick auf die internen Datenstrukturen gegeben,
%die hier eine Rolle spielen.
%Dieses Kapitel ist aufgeteilt in die verschiedenen Schritte, die bei der Verwendung der Bibliothek durchlaufen werden. Es
%wird jeweils darauf eingegangen, welche Änderungen notwendig sind und was jeweils zu beachten ist, um Typkonstruktorklassen
%zu unterstützen.


% - - - - - - - - - - - - - - - - - - - - - - - - - - - - - - - - - - - - - - - - - - - - - - - - - - - - - - - - - - - - - - - - - - - - - - - - - - - - - -
\subsection{Erweiterungen der BasicSyntax}
% - - - - - - - - - - - - - - - - - - - - - - - - - - - - - - - - - - - - - - - - - - - - - - - - - - - - - - - - - - - - - - - - - - - - - - - - - - - - - -

Alles beginnt wieder bei der Syntax. Der Sinn des eingeführten vereinfachten Syntaxbaums \texttt{BasicSyntax} war
die Einsparung nicht benötigter Sprachkonstrukte Haskells. Unter anderem ist dieser Einsparung die Applikation
von Typvariablen auf Typen zum Opfer gefallen, da diese ohne die Anwesenheit von Typkonstruktorklassen keine Rolle
spielte.

Der erste Schritt in der Erweiterung von \textit{free-theorems} ist also diese syntaktische Möglichkeit in der \texttt{BasicSyntax}
vorzusehen. Da \textit{free-theorems} intern Zugriff auf einen vollständigen Haskell-Parser hat, gestaltet sich
diese Änderung relativ einfach.
%Der Zweck der vereinfachten \texttt{BasicSyntax} war, dass nicht verwendete Sprachkonstrukte
%ignoriert werden können. Typvariablen, die auf andere Typen angewandt werden, gehörten bisher zu diesen vernachlässigbaren
%Konstrukten. Da im Hintergrund aber ein kompletter Haskell-Parser arbeitet, lässt sich diese Erweiterung relativ einfach
%wieder einpflegen.

Rein syntaktisch kann man Typkonstruktorvariablen als Typvariablen sehen, die wie Typkonstruktoren verwendet werden. Das
soll im folgenden Beispiel verdeutlicht werden.

%Bevor man mit Typkonstruktorvariablen arbeiten kann, müssen diese natürlich zunächst in der \texttt{BasicSyntax} vorgesehen
%sein. An sich verhalten sich diese ja wie andere Typvariablen auch, mit dem Unterschied, dass sie - ähnlich wie Funktionen -
%auf andere Typausdrücke angewandt werden können. Das folgende Beispiel soll dies verdeutlichen.

\begin{minted}{haskell}
(>>=) :: Monad m => m a -> (a -> m b) -> m b
\end{minted}

Das Beispiel zeigt die Signatur der Funktion \texttt{>>=}, in der die freie Typvariable \texttt{m} auf die Klasse \texttt{Monad}
eingeschränkt wird. Diese Klasse ist so definiert, dass sie einen Typparameter erwartet, der von der Sorte $* \rightarrow *$ ist,
folglich also selbst einen Parameter erwartet. Dementsprechend kommt in der Signatur beispielsweise der Ausdruck \texttt{m\ a} vor,
d.h. die Typkonstruktorvariable \texttt{m} wird angewandt auf die Typvariable \texttt{a}.

Es ist noch zu beachten, dass eine solche Applikation bei den Standardtypkonstruktoren bereits erlaubt ist.
So sind \texttt{[a]} und \texttt{Maybe a} auch ohne eine Anpassung möglich. Das liegt daran, dass bereits der
Haskell-Parser zwischen Standardkonstruktoren und Variablennamen unterscheidet -- Typvariablen werden durch \texttt{TyVar}
dargestellt, während für Standardkonstruktoren ein \texttt{TyCon}-Ausdruck im Syntaxbaum auftaucht.

%Da es sich
%bei \texttt{[]} und \texttt{Maybe} jedoch nicht um Variablen handelt, sondern um bekannte Typkonstruktoren, wird hier auch ein
%anderer Datentyp verwendet.

Um also dieser neuen syntaktischen Möglichkeit Rechnung zu tragen, wird die \texttt{BasicSyntax} um den Konstruktor
\texttt{TypeVarApp} erweitert. Am besten wird das deutlich, wenn man betrachtet, in welchen Ausdruck das obige Beispiel
nach der Anpassung in \texttt{BasicSyntax} überführt wird - siehe hierzu Listing \ref{lst:typevarapp}.

\begin{listing}[ht]
\begin{minted}{haskell}
(>>=) :: Monad m => m a -> (a -> m b) -> m b

TypeSig
(Signature {
   signatureName = (Ident "(>>=)"),
   signatureType =
      (TypeAbs (TV (Ident "a")) []
         (TypeAbs (TV (Ident "b")) []
            (TypeAbs (TV (Ident "m"))
               [(TypeClass (Ident "Monad"))]
               (TypeFun
                  (TypeVarApp (TV (Ident "m"))
                     [(TypeVar (TV (Ident "a")))])
                  (TypeFun
                     (TypeFun
                        (TypeVar (TV (Ident "a")))
                        (TypeVarApp (TV (Ident "m"))
                           [(TypeVar (TV (Ident "b")))]))
                     (TypeVarApp (TV (Ident "m"))
                        [(TypeVar (TV (Ident "b")))]))))))})
\end{minted}
\caption{Beispielsignatur von $>>=$ in BasicSyntax-Struktur}
\label{lst:typevarapp}
\end{listing}

Damit ist bereits die einzige benötigte Syntaxänderung eingeführt -- Klassendeklarationen sind in \textit{free-theorems}
ohnehin schon möglich. Natürlich sollte Typvariablenapplikation in jeder Typsignatur möglich sein, unabhängig davon, ob diese
Signatur eine Toplevel-Signatur ist oder ob sie sich innerhalb einer Klassendeklaration befindet.

Die einzige Änderung, die dazu nötig ist, findet in der Funktion \texttt{mkAppTyEx} statt, die für die Überführung einer
Typapplikation aus dem Originalsyntaxbaum in die \texttt{BasicSyntax} zuständig ist. Diese Funktion wird immer aufgerufen,
egal um welche Art von Typsignatur es sich handelt.
%Innerhalb der
%Klassendeklaration muss Variablenapplikation natürlich auch möglich sein. Das ist aber automatisch möglich, da sich die Änderungen
%auf sämtliche Funktionssignaturen erstrecken, unabhängig ob Toplevel-Signaturen oder Signaturen von Klassenfunktionen.

%Da Typvariablenapplikation in \cite{freetheorems} nicht vorgesehen ist, musste ein solches Konstrukt in die BasicSyntax-Struktur eingefügt werden. Hierbei ist zu beachten: Die Applikation
%von Standardtypkonstruktoren, beispielsweise \texttt{[a]} für Listen, ist sehr wohl vorgesehen und wird in der BasicSyntax durch \texttt{TypeCon} im Datentyp \texttt{TypeExpression} dargestellt. \todo{Wahrscheinlich viel zu speziell} Diese Standardtypkonstruktoren werden in diesem Zusammenhang aber als Sonderfälle betrachtet \ref{sec:freie-theoreme}, um das System wirklich um eigene Typklassen zu erweitern, muss es auch möglich sein, Typkonstruktorvariablen auf Typen anzuwenden.
%
%Die Erweiterung der BasicSyntax sieht hier so aus, dass ein neuer Konstruktor \texttt{TypeVarApp} zum Datentypen \texttt{TypeExpression} hinzugefügt wird. In Listing \ref{lst:typevarapp} ist ein Beispiel zu sehen.
%
%\inputminted[tabsize=2]{haskell}{typevarapp.hs}
%
%Abgesehen vom offensichtlichen Problem, dass dieser Konstruktor bei der Theoremgenerierung zu Fehlern beim Pattern Matching oder zumindest zu unerwünschtem Verhalten führt,
%ist auch zu beachten, dass die in Abschnitt \ref{sec:check} erwähnte \texttt{check}-Funktion entsprechend angepasst wird.


% - - - - - - - - - - - - - - - - - - - - - - - - - - - - - - - - - - - - - - - - - - - - - - - - - - - - - - - - - - - - - - - - - - - - - - - - - - - - - -
\subsection{Erweiterung des Relations-Datentyps}
% - - - - - - - - - - - - - - - - - - - - - - - - - - - - - - - - - - - - - - - - - - - - - - - - - - - - - - - - - - - - - - - - - - - - - - - - - - - - - -

% Da eine neue Relationskonstruktion eingeführt wird, muss auch der entsprechende Datentyp angepasst werden.
Der abstrakte Syntaxbaum in der \texttt{BasicSyntax}-Struktur wird von der \textit{interpret}-Funktion in eine relationale
Darstellung überführt. Das folgende Beispiel beinhaltet zur Verdeutlichung ein explizites \texttt{forall}.

\begin{minted}{haskell}
test :: forall a. a -> a
\end{minted}

Die Typsignatur dieses Beispiels lässt sich als Relation $\forall \mathcal{R}. \mathcal{R} \rightarrow \mathcal{R}$ auffassen.
Dabei ist $\forall \mathcal{R}. \mathcal{A}(\mathcal{R})$ so definiert, dass über alle Relationen aller Typmengen allquantifiziert
wird (vgl. Abschnitt \ref{sec:free-theorems-param}). \todo{richtig so?} So weit ist das Vorgehen bereits bekannt und auch schon
in \textit{free-theorems} implementiert.

Interessant ist nun, wie es sich verhält, wenn eine solche Typvariable auf Typausdrücke angewandt wird, wie das folgende
Beispiel zeigt.

\begin{minted}{haskell}
test :: forall f a. Functor f => f a -> f a
\end{minted}

Dieses Beispiel wird überführt in $\forall^{\{Functor\}} \mathcal{F} \forall \mathcal{R}. \mathcal{F} \mathcal{R} \rightarrow
\mathcal{F} \mathcal{R}$, wobei $\mathcal{F}$ \textit{keine} Typrelation ist. Es handelt sich hierbei, im Gegensatz zu $\mathcal{R}$ aus
dem ersten Beispiel, um eine Fuktion, die eine Relation auf eine andere Relation abbildet (natürlich wurde bereits festgestellt,
dass jede Funktion auch als Relation aufgefasst werden kann, das ist an dieser Stelle aber nicht gemeint). Es wird hier also auch nicht über
Relationen allquantifiziert, sondern über entsprechende Funktionen, was bedeutet, dass es sich um eine andere Konstruktion
handelt als die Allquantifizierung über einfache Typvariablen.

Aus diesem Grund wird im Relations-Datentyp \texttt{Relation} ein zusätzlicher Konstruktor eingeführt für die Typkonstruktorabstraktion, der
\texttt{RelTypeConsAbs} genannt wird. Außerdem wird der Konstruktor \texttt{RelTypeConsApp} eingeführt, der benutzt
wird, wenn man eine Relationsfunktion auf eine Relation anwendet, im obigen Beispiel also $\mathcal{F} \mathcal{R}$.

Tabellen \ref{tab:reltypeconsabs} und \ref{tab:reltypeconsapp} zeigen die Parameter für diese neuen Datentypkonstruktoren.

\begin{table}[th]
\begin{tabular}{ | l | l | }
\hline
RelTypeConsAbs & \\
\hline
\texttt{RelationInfo} & Zusätzliche Informationen zur Relation, in jedem Konstruktor \\
& enthalten. \\
\texttt{RelationVariable} & Funktionsvariable, die allquantifiziert wird. \\
\texttt{(TypeExpression,} & Typkonstruktorvariablen auf der linken und rechten Seite. \\
\texttt{TypeExpression)} & \\
\texttt{[Restriction]} & Einschränkungen auf Klassen, aber auch auf \textit{Striktheit},\\
& \textit{Stetigkeit}, etc. \\
\texttt{Relation} & Die Relation, die die quantifizierte Variable enthält. \\
\hline
\end{tabular}
\caption{Parameter des Konstruktors \texttt{RelConsAbs}}
\label{tab:reltypeconsabs}
\end{table}

\begin{table}[th]
\begin{tabular}{ | l | l | }
\hline
RelTypeConsApp & \\
\hline
\texttt{RelationInfo} & Zusätzliche Informationen zur Relation, in jedem Konstruktor \\
& enthalten. \\
\texttt{RelationVariable} & Relationsfunktionsvariable, die auf eine Relation angewandt\\
& wird. \\
\texttt{Relation} & Relation, auf die die Typkonstruktorfunktion angewandt wird. \\
\hline
\end{tabular}
\caption{Parameterliste des Konstruktors \texttt{RelConsApp}}
\label{tab:reltypeconsapp}
\end{table}

Die meisten Parameter sollten selbsterklärend sein. Die Bedeutung der \texttt{RelationInfo}-Struktur wurde ja bereits in Abschnitt \ref{sec:free-theorems-interpret} erläutert \todo{das fehlt noch}. Wichtig ist hier, dass in \texttt{RelTypeConsAbs} das
Tupel von \texttt{TypeExpressions} keine Typvariablen sind, sondern dass es sich hierbei um Variablen für Typkonstruktoren handelt.

Mit \texttt{RelConsFunVar} wird noch ein zusätzlicher Konstruktor für den \texttt{Relation}-Datentyp eingeführt. Dieser hat
die gleichen Parameter wie der \texttt{RelVar}-Datentyp und wird auch nicht in die \texttt{Intermediate}-Struktur eingebaut.
Er dient als Hilfsmittel für den Interpretationsschritt, der in Abschnitt \ref{sec:extend-interpret} erklärt wird.

Mit diesen Erweiterungen ist es nun möglich, die hinzukommenden relationalen Strukturen auszudrücken.


% - - - - - - - - - - - - - - - - - - - - - - - - - - - - - - - - - - - - - - - - - - - - - - - - - - - - - - - - - - - - - - - - - - - - - - - - - - - - - -
\subsection{Zusätzliche Fehlerprüfungen}
% - - - - - - - - - - - - - - - - - - - - - - - - - - - - - - - - - - - - - - - - - - - - - - - - - - - - - - - - - - - - - - - - - - - - - - - - - - - - - -

\label{sec:erweiterung-typklassen-fehler}

%Da in der ursprünglichen Version von free-theorems Typkonstruktorvariablen nicht gestattet waren, konnten einige
%Fehlerüberprüfungen eingespart werden. Erlaubt man Typkonstruktorvariablen, muss man zusätzliche Fehlerfälle abdecken,
%die auftreten können. Betrachten wir das folgende Beispiel:

Versucht man in der ursprünglichen Version von \textit{free-theorems}, Typkonstruktorvariablen zu verwenden und auf
andere Typen zu applizieren, wird dies noch bei der Überführung in die \texttt{BasicSyntax} mit einer Fehlermeldung abgebrochen.
Dadurch fallen natürlich ein paar Fehlerfälle weg, die erst durch Typkonstruktorvariablen auftreten können. Gestattet man
die Benutzung von Typkonstruktorvariablen und -klassen, muss man auch diese Probleme beachten.

%In der ursprünglichen Version von \textit{free-theorems} können Typkonstruktorvariablen nicht verwendet werden. Es wird eine
%Fehlermeldung ausgegeben, sobald eine Typvariable auf eine andere angewandt wird. Gestattet man dies, müssen neu
%entstehende Problemfälle abgedeckt werden.

Das folgende Beispiel zeigt ein solches Problem.

\begin{minted}{haskell}
appl :: Functor f => f a -> f
\end{minted}

Hierbei handelt es sich nicht um eine korrekte Haskell-Typsignatur: Die als \texttt{Functor} eingeführte Variable \texttt{f} wird in der Signatur
einmal mit einem Parameter, einmal ohne Parameter aufgerufen. Selbst wenn man die Deklaration von \texttt{Functor}
außer Acht lässt, kann man deutlich sehen, dass eines der beiden Vorkommen von \texttt{f} fehlerhaft sein muss. \texttt{f} erwartet
entweder einen oder gar keinen Parameter, unterschiedliche Parameterzahlen sind nicht gestattet.

Da Functor bekannt ist, kann man einfach die Deklaration dieser Klasse betrachten.

\begin{minted}{haskell}
class Functor f where
   fmap :: (a -> b) -> f a -> f b
\end{minted}

Anhand der Signatur der Klassenfunktion kann man nun sehen, dass die Variable \texttt{f} stets auf einen einzelnen 
Parameter angewandt wird, das heißt, man kann bei allen Vorkommen der Klasse \texttt{Functor} davon ausgehen, dass diese
auf genau einen Parameter appliziert werden müssen. Natürlich kommt hier eine weitere Überprüfung ins Spiel: Es muss
sichergestellt werden, dass die Klassenvariable -- in diesem Fall \texttt{f} -- in sämtlichen Klassenfunktionen mit der gleichen Anzahl
an Parametern verwendet wird.

Die Überprüfung, in der sichergestellt wird, dass die Klassenvariable einer Typklasse innerhalb dieser Klasse immer mit
der gleichen Anzahl an Parametern verwendet wird, ist eine lokale Überprüfung --
%Die Überprüfung der Klassenvariable innerhalb ihrer eigenen Typklasse auf die korrekte Parameteranzahl aller Vorkommen der Klassenvariable ist eine lokale Überprüfung --
es wird stets nur die Klassendeklaration benötigt, ein globaler Kontext muss nicht bekannt sein. Möchte man überprüfen,
ob Typkonstruktorvariablen in beliebigen Deklarationen die korrekte Anzahl an Parametern haben, dann kommt man um den
globalen Kontext nicht herum, da es aus jeder Signatur heraus möglich sein muss, die Klassendeklaration einzusehen. Letzteres
fällt also in die Kategorie der globalen Überprüfungen.
%Die Deklaration der Klasse muss in Betracht gezogen werden.

Eine Besonderheit ist hierbei auch, dass eine Typvariable auf mehrere Klassen gleichzeitig eingeschränkt werden kann. So ist
das folgende Beispiel legitimer Haksell-Code:

\begin{minted}{haskell}
convert :: (SomeClass1 f, SomeClass2 f) => f a -> f b
\end{minted}

%Man muss also zusätzlich sicherstellen, dass alle Klassen, die für die gleichen Variablen angegeben sind, die gleiche Anzahl an
%Parametern erwarten.

Es muss also sichergestellt werden, dass alle Klassen, die pro Variable angegeben sind, die gleiche Anzahl an Parametern erwarten.
Dieses Problem erledigt sich allerdings automatisch, wenn man die rechte Seite auf Fehler überprüft, da im Falle unterschiedlicher
Aritäten mindestens einer der beiden Ausdrücke \texttt{f a} und \texttt{f b} einen Fehler erzeugen würde.


% - - - - - - - - - - - - - - - - - - - - - - - - - - - - - - - - - - - - - - - - - - - - - - - - - - - - - - - - - - - - - - - - - - - - - - - - - - - - - -
\subsection{Erweiterung der Typinterpretation}
% - - - - - - - - - - - - - - - - - - - - - - - - - - - - - - - - - - - - - - - - - - - - - - - - - - - - - - - - - - - - - - - - - - - - - - - - - - - - - -

\label{sec:extend-interpret}

%TODO: erklären, dass ein zusätzlicher Relationskonstruktor erzeugt wurde.

Die \texttt{interpret}-Funktion ist dafür zuständig, den abstrakten Syntaxbaum von Typ\-sig\-na\-tu\-ren in die enstprechende
\texttt{Intermediate}-Struktur zu überführen. Natürlich sind hier Änderungen notwendig, wenn \texttt{interpret} auch
Typkonsturktorvariablen und deren Applikation erkennen und korrekt behandeln soll.

Es wurde ja bereits in Abschnitt \ref{sec:free-theorems-interpret} erläutert, wie die Funktion mit dem \texttt{forall}-Konstrukt
umgeht: Es wird in eine Allquantifizierung über Relationen umgewandelt, also in einen \texttt{RelAbs}-Ausdruck.
Da es sich hierbei nicht mehr zwingend um Relationen handelt, wenn Typkonstruktorvariablen vorkommen können, sondern auch
Funktionen auf Relationen afutreten können, wird jetzt an dieser Stelle eine Fallunterscheidung gemacht.

In Abschnitt \ref{sec:free-theorems-interpret} wurde auch beschrieben, dass die Implementierung von \textit{free-theorems} hier eine
\texttt{Map} als Umgebung nutzt, in der Typvariablen den dazugehörigen Variablen-Ausdrücken zugeordnet werden. \todo{wurde es?}

Dieser Vorgang wird nun aufgeteilt. Es wird zunächst überprüft, ob die Typvariable, über die allquantifiziert wird, in der
restlichen Typsignatur auf Parameter angewandt wird. Wird ein solches Vorkommen nicht gefunden, wird wie üblich vorgegangen
und es wird eine \texttt{RelAbs}-Struktur erzeugt; zudem wird in der angesprochenen $Map$ dem Namen der Typvariablen
die erzeugte Relationsvariablen-Struktur \texttt{RelVar} zugeordnet.

Wird stattdessen eine Applikation der Typvariablen gefunden, wird diese im Folgenden als Typkonstruktorvariable angesehen,
und es wird statt \texttt{RelAbs} ein \texttt{RelTypeConsAbs}-Ausdruck erzeugt.

%Es wird überprüft, ob die entsprechende Typvariable, über die allquantifiziert wird, in der restlichen Typsignatur auf Parameter
%angewandt wird. Gibt es irgendwo ein solches Vorkommen, wird statt der \texttt{RelAbs} die neue \texttt{RelTypeConsAbs}-Struktur
%verwendet.
Von den Parametern her unterscheiden sich die Konstruktoren nicht. Der \texttt{RelTypeConsAbs}-Konstruktor wird nur
eingeführt, damit die entsprechende Typkonstruktorvariable nicht wie normale Typvariablen verwendet wird.
Zum Beispiel macht es keinen Sinn, die Relationsfunktionen zu Typkonstruktorvariablen auf
Funktionen zu spezialisieren. Auch sieht natürlich die resultierende Formel für Typkonstruktorabstraktion anders aus als die
für Typabstraktion.

Ein weiterer Unterschied besteht im Ausdruck, der in die Umgebungs-$Map$ eingetragen wird. Während es sich bei
Typvariablen um \texttt{RelVar}-Ausdrücke handelt, die dann später bei jedem Auftreten der Typvariable im Syntaxbaum
in die Relationaldarstellung eingesetzt wird, wird bei Typkonstruktorvariablen eine neu eingeführte
\texttt{RelConsFunVar}-Struktur eingetragen.
Dies dient dazu, auch beim Auftreten der Typvariablen sicherzustellen, dass es sich nicht wirklich um eine
Typkonstruktorvariable handelt.

Da in \texttt{check} bereits abgefangen wird, wenn Typkonstruktorvariablen mit der falschen Zahl an Parametern aufgerufen
werden, kann in \texttt{interpret} davon ausgegangen werden, dass es sich bei einem Vorkommen einer
Applikation von Typvariablen automatisch um eine Typkonstruktorvariable handeln muss.



% - - - - - - - - - - - - - - - - - - - - - - - - - - - - - - - - - - - - - - - - - - - - - - - - - - - - - - - - - - - - - - - - - - - - - - - - - - - - - -
\subsection{Erweiterung der Spezialisierung auf Funktionen}
% - - - - - - - - - - - - - - - - - - - - - - - - - - - - - - - - - - - - - - - - - - - - - - - - - - - - - - - - - - - - - - - - - - - - - - - - - - - - - -

Durch Anwenden der Funktion \textit{specialise} wird die Formel des Theorems teilweise stark vereinfacht, da relationale
Aussagen der Form $(x, y) \in \mathcal{R}$ in Gleichungen der Form $f\ x = y$ überführt werden. Es wäre wünschenswert,
diese Vereinfachungen auch für Ausdrücke mit Typkonstruktorvariablen nutzbar zu machen.

In Abschnitt \ref{sec:specialise-relvars} wurde bereits beschrieben, wie \textit{specialise} in \textit{free-theorems} bisher arbeitete:
Im ersten Schritt werden alle Vorkommen der zu spezialisierenden Relationsvariable durch eine Funktionsvariable ersetzt,
im zweiten Schritt werden geliftete Relationen dort, wo es möglich ist, durch spezielle Funktionen wie $map$ und $fmap$ ersetzt.
Es liegt die Vorgehensweise nahe, die bisherige Methode auch auf beliebige Typkonstruktorvariablen auszuweiten. Hier tritt nur leider
das Problem auf, das bereits in Abschnitt \ref{sec:freie-theoreme-anwendung} angedeutet wurde: Man hat bei Allquantifizierung
über Typkonstruktoren keine Informationen über die konkret verwendete
Datenstruktur. Alles, was bekannt ist, ist die verwendete Typklasse, die die Existenz gewisser Funktionen voraussetzt.

Doch selbst wenn diese Klassenfunktionen gegeben sind, kann man keine Aussage darüber treffen, was diese Funktionen tun, da
Typklassen Ad-Hoc-Polymorphismus nutzen: Über die tatsächliche Implementierung ist nichts bekannt. Das heißt, dass man selbst im Falle, dass
die Relationsfunktionsvariable $\mathcal{F}$ auf Instanzen der Typklasse \texttt{Functor} eingeschränkt ist, nicht darauf schließen kann, dass ein Ausdruck der Art $(x, y) \in \mathcal{F}\ \mathcal{R}$ auf den Ausdruck
$fmap\ f\ x = y$ spezialisiert werden kann.

%Eine erste Implementierung sah diesen Ansatz vor, wurde dann aber wieder entfernt.
Eine erste Implementierung sah die Verwendung unterschiedlicher spezieller Funktionen vor, abhängig von der Typklasse der
Typkonstruktorvariable. Das folgende Beispiel zeigt das freie Theorem für die Typsignatur \texttt{Functor f => a -> f a}.

\begin{align*}
& \forall k_1, k_2 \in (* \rightarrow *), \mathcal{K} : k_1 \Leftrightarrow k_2, \mathcal{K}\ \text{beachtet Functor} \\
& \forall T_1, T_2 \in Typen, \mathcal{R} : T_1 \Leftrightarrow T_2 \\
& \forall (x, y) \in \mathcal{R} . (test_{t_1\ k_1}\ x, test_{t_2\ k_2}\ y) \in \mathcal{K}\ \mathcal{R}
\end{align*}

Spezialisiert man in diesem Beispiel die Relation $\mathcal{R}$ auf eine Funktion $f$, kann man auf die Idee kommen, dieses Theorem
wie folgt zu transformieren.

\begin{align*}
& \forall k_1, k_2 \in (* \rightarrow *), \mathcal{K} : k_1 \Leftrightarrow k_2, \mathcal{K}\ \text{beachtet Functor} \\
& \forall T_1, T_2 \in Typen, f : T_1 \rightarrow T_2 \\
& \forall x \in T_1 . \text{fmap}\ f\ (test_{t_1 k_1}\ x) = test_{t_2 k_2} (f\ x)
\end{align*}
% TODO: Was ist, wenn Functor-Gesetze gelten?

Aus den oben genannten Gründen muss dies aber nicht immer gelten. Die jeweilige \texttt{fmap}-Funktion könnte beispielsweise
die Reihenfolge der Daten ändern, die übergebene Funktion ignorieren, etc.
Aus diesem Grund wird die Zeile transformiert wie folgt:

\begin{align*}
& \forall x \in t_1 . (test_{t_1 k_1}\ x, test_{t_2 k_2}\ (f\ x)) \in \mathcal{K}\ f
\end{align*}

Da Funktionen spezielle Relationen sind, kann man umgekehrt aus der Funktion $f$ wiederum eine Relation erzeugen, indem
man die Relationsfunktion $\mathcal{K}$ auf $f$ anwendet. Es ist natürlich unglücklich, dass dadurch die Spezialisierung
der Relation prinzipiell wieder ``umgekehrt'' wird und wieder nur eine Aussage über Relationen getroffen wird, andererseits
kann man nicht davon ausgehen, dass jede Funktion, die durch die Relationsfunktion transformiert wird, wieder eine Funktion ist;
man muss also bei der Relationaldarstellung bleiben.

%Hierbei ist $lift_{\mathcal{K}}(f)$ die Relation, die durch das \textit{Lifting} der Funktion $f$ in den $\mathcal{K}$-Kontext
%entsteht. Das ist in diesem Fall notwendig, da $\mathcal{K}$ als Funktion auf Relationen eingeführt wurde. Zwar werden
%Funktionen hier als spezielle Relationen behandelt, rein formal betrachtet kann man an dieser Stelle jedoch nicht
%einfach die Relationsfunktion $\mathcal{K}$ auf eine Funktion $f$ anwenden.

%Die \texttt{specialise}-Funktion überführt an dieser Stelle die \texttt{RelTypeConsApp}-Struktur in eine \texttt{RelLift}-Struktur,
%die eine geliftete Relation darstellt. Normalerweise wird diese Struktur mit einer Angabe darüber gefüllt, um welchen speziellen
%Typkonstruktor es sich handelt, beispielsweise der Listenkonstruktor, der Tupelkonstruktor oder einfach der Name eines
%definierten Datentyps.

%Für den Fall einer Typkonstruktorvariable wurde \texttt{ConVar} als weitere Möglichkeit eingeführt. Diesem Konstruktor wird
%der Name der Typkonstruktorvariablen übergeben \todo{ist das nicht auch einfach nur die relationsfunktion, angewandt auf relation?}


% - - - - - - - - - - - - - - - - - - - - - - - - - - - - - - - - - - - - - - - - - - - - - - - - - - - - - - - - - - - - - - - - - - - - - - - - - - - - - -
\subsection{Erweiterung der Theoremgenerierung}
% - - - - - - - - - - - - - - - - - - - - - - - - - - - - - - - - - - - - - - - - - - - - - - - - - - - - - - - - - - - - - - - - - - - - - - - - - - - - - -

Die Funktion \texttt{asTheorem} wandelt die \texttt{Intermediate}-Darstellung um in eine Formel. Natürlich sind auch hier
Anpassungen vonnöten, um auf vorkommende Typkonstruktorvariablen zu reagieren. Der Datentyp, mit dem Formeln
dargestellt werden, ist \texttt{Formula}. Auch dieser muss um einen Konstruktor für die Allquantifizierung über
Typkonstruktorvariablen erweitert werden, \texttt{ForallTypeConstructors} genannt.
Dieser Konstruktor ist definiert wie der Konstruktor \texttt{ForallRelations}, nur wird er verwendet für die Allquantifizierung
über Relationsfunktionen.

Es gibt zwei neue Fälle zu beachten, was die Transformation von Relationen in Formeln angeht: Zum einen müssen die
entsprechenden Relationsabstraktionen in die neue \texttt{ForallTypeConstructors}-Struktur überführt werden, zum anderen
müssen die Vorkommen von \texttt{RelTypeConsApp} behandelt werden.

%Ähnlich wie auch bei anderen Typkonstruktoren, werden angewandte Typkonstruktorvariablen in einen \textit{lift}-Ausdruck
%überführt. Folgender Ausdruck sei als Beispiel gegeben.

\begin{align*}
\forall^{\{Functor\}} \mathcal{F} \forall \mathcal{R} . \mathcal{R} \rightarrow \mathcal{F} \mathcal{R}
\end{align*}

Der Ausdruck $\mathcal{F} \mathcal{R}$ aus diesem Beispiel wird als \texttt{Relation}-Struktur dargestellt durch den folgenden
(hier leicht vereinfachten) Haskell-Ausdruck.

\begin{minted}{haskell}
(RelTypeConsApp ("k1 t1", "k2 t2") "F" (RelVar "R"))
\end{minted}

%Die Typausdrücke in Klammern sind die Typen für den gesamten Ausdruck. Zwar handelt es sich hierbei mit \texttt{TypeExpression}
%um einen \texttt{BasicSyntax}-Datentyp, dieser enthält jedoch einen speziellen Typkonstruktor namens \texttt{TypeExp}, mit
%dem beliebige Typen als Zeichenketten dargestellt werden können.

Ähnlich wie Datentypen, die auf eine Typvariable angewandt werden, wird auch die Typkonstruktorvariablenapplikation
behandelt. Zur Erinnerung: Datentypkonstruktoren, angewandt auf allquantifizierte Typvariablen, werden durch eine
\textit{lift}-Funktion als Relation ausgedrückt, deren Definition durch die \texttt{unfoldLifts}-Funktion erzeugt werden kann.

Bei Typkonstruktorvariablen wird eine solche explizite Lifting-Funktion nicht benötigt, da sie nicht als Relationen repräsentiert
werden, sondern als Funktionen auf Relationen. Da keine weiteren Aussagen zu einem Ausdruck wie $\mathcal{F}\ \mathcal{R}$
gemacht werden können, wird dieser Ausdruck bei Anwendung von \texttt{asTheorem} beibehalten.

Da die \texttt{specialise}-Funktion in Bezug auf Typkonstruktorabstraktion und -applikation keine Auswirkungen hat, wie im
vorangegangenen Abschnitt erläutert wurde, hat dies auch keine Auswirkungen auf die \texttt{asTheorem}-Funktion.

%Im vorangegangenen Abschnitt wurde gezeigt, dass bei Anwendung von \texttt{specialise} Funktionsvariablenapplikation
%durch 

%Wie im vorangegangenen Abschnitt bereits erläutert wurde, wird der \texttt{RelTypeConsApp}-Konstruktor bei einer Spezialisierung
%zu einem \texttt{RelLift}-Ausdruck, sofern er die entsprechende Relationsvariable beinhaltet, auf die spezialisiert wird.

% TODO: noch ein bisschen genauer!

% - - - - - - - - - - - - - - - - - - - - - - - - - - - - - - - - - - - - - - - - - - - - - - - - - - - - - - - - - - - - - - - - - - - - - - - - - - - - - -
\subsection{Beispieldurchlauf}
% - - - - - - - - - - - - - - - - - - - - - - - - - - - - - - - - - - - - - - - - - - - - - - - - - - - - - - - - - - - - - - - - - - - - - - - - - - - - - -

Es wurden nun sämtliche Modifikationen an der Bibliothek erläutert, doch um das Zusammenspiel der einzelnen Änderungen
einmal im Zusammenhang zu sehen, wird in diesem Abschnitt ein kompletter Durchlauf für eine Signatur dargestellt, die eine
Typkonstruktorvariable beinhaltet.

%Um die Funktionsweise der Erweiterungen zu verdeutlichen und einen Einblick in die Änderungen zu geben, wird in diesem Abschnitt
%ein kompletter Durchlauf der erweiterten Bibliothek erläutert, wobei sämtliche Zwischendarstellungen der Daten angegeben werden.

Wir verwenden das Beispiel, das bereits in Abschnitt \ref{sec:typkonstruktorklassen} betrachtet wurde.

\begin{minted}{haskell}
fmap :: Functor f => (a -> b) -> f a -> f b
\end{minted}

Dieser Ausdruck wird durch \texttt{parse} in die \texttt{BasicSyntax}-Struktur überführt und es ergibt sich, vereinfacht dargestellt,
der Ausdruck in Listing \ref{lst:fmap-ast}.

\begin{listing}[ht]
\begin{minted}{haskell}
(TypeSig
   (Signature {
      signatureName = (Ident "fmap"),
      signatureType =
         (TypeAbs (TV (Ident "a"))
            []
            (TypeAbs (TV (Ident "b"))
               []
               (TypeAbs (TV (Ident "f"))
                  [(TypeClass (Ident "Functor"))]
                  (TypeFun
                     (TypeFun
                        (TypeVar (TV (Ident "a")))
                        (TypeVar (TV (Ident "b")))
                     )
                     (TypeFun
                        (TypeVarApp (TV (Ident "f"))
                         [(TypeVar (TV (Ident "a")))])
                        (TypeVarApp (TV (Ident "f"))
                        [(TypeVar (TV (Ident "b")))]))
                     ))))}))
\end{minted}
\caption{BasicSyntax-Syntaxbaum für das fmap-Beispiel}
\label{lst:fmap-ast}
\end{listing}

Man kann den neu hinzgekommenen Konstruktor \texttt{TypeVarApp} sehen, der eine Liste von Ausdrücken erwartet, auf
die die entsprechende Typvariable angewandt wird. Zu beachten ist auch, dass auf Syntax-Ebene sowohl Typvariablen als auch
Typkonstruktorvariablen über denselben Ausdruck \texttt{TypeAbs} allquantifiziert werden. Unterschieden wird hier erst im
nächsten Schritt.

Die Liste, die der \texttt{TypeAbs}-Konstruktor als zweiten Parameter erwartet, ist die Liste der Einschränkungen -- auf Syntaxebene
sind dies die im Kontext angegebenen Typklassen, in diesem Beispiel also die Klasse \texttt{Functor}.

Der nächste Schritt, der Aufruf der \texttt{check}-Funktion, soll nur am Rande erwähnt bleiben, da die entsprechenden
Fehlerüberprüfungen in Abschnitt \ref{sec:erweiterung-typklassen-fehler} bereits erläutert wurden und in unserem Beispiel
kein Fehler zu erwarten ist -- zudem ändert sich die Datenstruktur dadurch kaum.

Wichtig ist aber zu erwähnen, dass die \texttt{check}-Funktion (bzw. die \texttt{checkAgainst}) natürlich sicherstellt, dass die
\texttt{Functor}-Klasse deklariert ist.
In unserem Beispiel ist die eigentliche Deklaration der Klasse nicht enthalten. Natürlich muss die \texttt{check}- bzw später
auch die \texttt{interpret}-Funktion wissen, welche vordefinierten Funktionen,
Datentypen und Klassen gegeben sind. In der Implementierung des Webinterface \cite{freetheoremswebui} werden zum Beispiel bereits
geparste \texttt{BasicSyntax}-Ausdrücke vieler Deklarationen der Prelude aus einem separaten Modul importiert und den
Funktionen \texttt{checkAgainst} und \texttt{interpret} zusätzlich zur Hauptsignatur übergeben, sodass sie dieser bekannt sind.
Außerdem wird ein Eingabefeld angezeigt, in dem es möglich ist, zusätzliche Deklarationen anzugeben.

Wir gehen im Folgenden davon aus, dass die typische Deklaration der \texttt{Functor}-Klasse gegeben ist, wie sie in dieser
Arbeit auch schon mehrfach verwendet wurde. Der nächste Schritt besteht wieder darin, die Funktion $interpret$ aufzurufen. 
Diese Funktion überführt die \texttt{BasicSyntax} nun also in die \texttt{Intermediate}-Darstellung, die vereinfacht in Listing
\ref{lst:example-intermediate} zu sehen ist.

\begin{listing}[ht]
\begin{minted}{haskell}
(Intermediate "fmap" BasicSubset
   (RelAbs "R" ("t1", "t2")
      (RelAbs "S" ("t3", "t4")
         (RelTypeConsAbs ("k1", "k2")
            (RelFun
               (RelFun
                  (RelVar "R")
                  (RelVar "S"))
                  (RelFun
                     (RelTypeConsApp ("k1 t1", "k2 t2") "R1" (RelVar "R"))
                     (RelTypeConsApp ("k1 t3", "k2 t4") "R1" (RelVar "S"))
                  ))))) ...)
\end{minted}
\caption{Intermediate-Darstellung des Beispiels}
\label{lst:example-intermediate}
\end{listing}

Dieser Schritt ist wieder relativ selbsterklärend, da lediglich jede Typkonstruktion in die entsprechende Relationskonstruktion
überführt wird. Die \texttt{RelTypeConsApp}-Ausdrücke entsprechen dabei der Applikation einer Relationsfunktion auf
eine Relation. Die Typen sind dabei die Typkonstruktoren, angewandt auf die Typvariablen.

An dieser Stelle liegt also eine \texttt{Intermediate}-Struktur vor, die die Relationaldarstellung der Typsignatur enthält. Als
nächstes soll diese Relationaldarstellung spezialisiert werden auf Funktionen, es wird als zunächst durch einen Aufruf von
\texttt{relationVariables} die Liste aller Relationsvariablen ermittelt, für jede dieser Variablen wird dann \texttt{specialise}
aufgerfen.

Die \texttt{Intermediate}-Struktur sieht daraufhin so aus wie in Listing \ref{lst:example-intermediate-specialised}.

\begin{listing}[ht]
\begin{minted}{haskell}
(Intermediate "fmap" BasicSubset
   (FunAbs "f" ("t1", "t2")
      (FunAbs "g" ("t3", "t4")
         (RelTypeConsAbs ("k1", "k2")
            (RelFun
               (RelFun
                  (FunVar "f")
                  (FunVar "g"))
               (RelFun
                  (RelTypeConsApp ("k1 t1", "k2 t2") "R1"
                     (FunVar "f"))
                  (RelTypeConsApp ("k1 t3", "k2 t4") "R1"
                     (FunVar "g"))))))) ...)
\end{minted}
\caption{Spezialisierte Intermediate-Darstellung des Beispiels}
\label{lst:example-intermediate-specialised}
\end{listing}

Wie man sieht, werden sämtliche Vorkommen von Relationsvariablen durch entsprechende Funktionsvariablen ersetzt, auch
die Variablen, die innerhalb der \texttt{RelTypeConsApp}-Struktur stehen. An den Applikationsausdrücken selbst hat sich
nichts geändert. Zu beachten ist noch, dass die als Tupel dargestellten Zeichenketten in \texttt{RelTypeConsApp} nur
der Einfachheit in dieser Form zu sehen sind -- tatsächlich handelt es sich jeweils um \texttt{RelationInfo}-Ausdrücke, in
allgemein in jedem Ausdruck des \texttt{Relation}-Datentyp vorkommen und den jeweils dargestellten Typ beinhalten.

An dieser Stelle kann also die Umwandlung in die Formel des freine Theorems stattfinden, wozu die Funktion \texttt{asTheorem}
bemüht wird. Diese Funktion führt nun das Abrollen der jeweiligen Relationsdefinitionen durch und erzeugt eine
entsprechende \texttt{Formula}-Struktur. Auch diese Zwischendarstellung soll an dieser Stelle nicht noch einmal thematisiert
werden, das Interessante ist die durch den Pretty Printer resultierende Zeichenkette, die wie folgt aussieht.

\begin{minted}{haskell}
"forall t1,t2 in TYPES, f :: t1 -> t2.
 forall t3,t4 in TYPES, g :: t3 -> t4.
  forall k1,k2 in (* -> *), R1 : k1 <=> k2, R1 respects Functor.
   forall p :: t1 -> t3.
    forall q :: t2 -> t4.
     (forall x :: t1. g (p x) = q (f x))
     ==> (forall (y, z) in R1 f.
           (fmap_{t1}_{t3}_{k1} p y, fmap_{t2}_{t4}_{k2} q z) in R1 g)"
\end{minted}

Und es zeigt sich, dass diese Zeichenkette dem in Abschnitt \texttt{sec:freie-theoreme-anwendung} per Hand hergeleiteten
freien Theorem zu \texttt{fmap} entspricht.
% TODO: Beispieldurchlauf zuende schreiben