% % % % % % % % % % % % % % % % % % % % % % % % % % % % % % % % % % % % % % % % %
\section{Anhang}
% % % % % % % % % % % % % % % % % % % % % % % % % % % % % % % % % % % % % % % % %

% - - - - - - - - - - - - - - - - - - - - - - - - - - - - - - - - - - - - - - - - - - - - - - - - - - - - - - - - - - - - - - - - - - - - - - - - - - - - - -
\subsection*{Übersicht}
% - - - - - - - - - - - - - - - - - - - - - - - - - - - - - - - - - - - - - - - - - - - - - - - - - - - - - - - - - - - - - - - - - - - - - - - - - - - - - -

Die in dieser Arbeit angesprochene Erweiterung der Bibliothek \textit{free-theorems} liegt in der folgenden
Ordnerstruktur vor. Dabei handelt es sich sowohl um das angepasste Paket \textit{free-theorems} als auch um
eine minimal angepasste Version von \textit{free-theorems-webui} und ein Testwerkzeug in \texttt{test-free-theorems}.
\\
\dirtree{%
.1 free-theorems-0.3.2.0.
.2 src.
.3 Language.
.4 Haskell.
.5 FreeTheorems.
.6 Frontend.
.6 Parser.
.6 Theorems.
.1 free-theorems-webui-0.2.1.1.
.1 test-free-theorems.
}

%Das erweiterte Paket liegt in der angegebenen Ordnerstruktur vor. Dabei enthält \texttt{free-theorems-0.3.2.0} die
%um Typkonstruktorklassen erweiterte Bibliothek \textit{free-theorems}. Die Erweiterung baut auf Version 0.3.2.0
%des Originalpakets auf.


% - - - - - - - - - - - - - - - - - - - - - - - - - - - - - - - - - - - - - - - - - - - - - - - - - - - - - - - - - - - - - - - - - - - - - - - - - - - - - -
\subsection*{Installation}
% - - - - - - - - - - - - - - - - - - - - - - - - - - - - - - - - - - - - - - - - - - - - - - - - - - - - - - - - - - - - - - - - - - - - - - - - - - - - - -

Um das Paket aus dem Verzeichnis zu installieren, kann das Kommandozeilentool \texttt{cabal} verwendet werden,
da es sich bei \textit{free-theorems} um ein cabal-Paket handelt. Dieses Paket wurde bei der Erweiterung um
Typkonstruktorklassen angepasst, um mit neueren Versionen des GHC kompatibel zu sein, da die letzte Version
von \textit{free-theorems} etwas veraltet war. So wurden Versionsnummern der eingebundenen Pakete
sowie verwendete Spracherweiterungen angepasst.

Um das Paket zu installieren, reicht der folgende Aufruf im Verzeichnis namens \texttt{free-theorems-0.3.2.0}, in dem sich
auch die Datei \texttt{free-theorems.cabal} befindet.

\begin{minted}{text}
cabal install
\end{minted}

Das Paket wurde mit dem Glasgow Haskell Compiler Version 7.10.3 getestet und sollte auch mit neuren Versionen
funktionieren. Die vom Paket benötigten Abhängigkeiten sollten vom cabal-Tool selbstständig aufgelöst werden.

Um eine modifizierte Version des Weboberflächen-Pakets zu installieren, die zusätzlich mit den Typkonstruktorklassen
\texttt{Monad} und \texttt{Functor} ausgestattet ist, reicht ebenfalls der Befehl \texttt{cabal install} im Verzeichnis
\texttt{free-theorems-webui-0.2.1.1}. Tatsächlich handelt es sich dabei um das Originalpaket, in dem lediglich die
Datei \texttt{KnownDeclarations.hs} leicht modifiziert wurde. Die Bibliothek \textit{free-theorems} ist auch mit dem
unangepassten Original-Paket kompatibel, das sich über den folgenden Befehl installieren lässt.

\begin{minted}{text}
cabal install free-theorems-webui
\end{minted}

In diesem Fall unterstützt die Weboberfläche die Typklassen \texttt{Monad} und \texttt{Functor} nicht,
die grundsätzliche Funktionalität von Typkonstruktorklassen ist aber enthalten (es lassen sich beispielsweise
Typkonstruktorklassen im entsprechenden Feld selbst deklarieren).


% - - - - - - - - - - - - - - - - - - - - - - - - - - - - - - - - - - - - - - - - - - - - - - - - - - - - - - - - - - - - - - - - - - - - - - - - - - - - - -
\subsection*{Verwendung}
% - - - - - - - - - - - - - - - - - - - - - - - - - - - - - - - - - - - - - - - - - - - - - - - - - - - - - - - - - - - - - - - - - - - - - - - - - - - - - -

Sobald die Bibliothek installiert wurde, lässt sich \texttt{Language.Haskell.FreeTheorems} in eigenen
Haskell-Modulen importieren. Eine Beispielanwendung ist im Verzeichnis namens \texttt{test-free-theorems} zu finden in der
Datei \texttt{test.hs}. Diese wurde während der Erweiterung der Bibliothek hauptsächlich zum Testen entwickelt,
liefert jedoch die meisten Funktionalitäten, die die Bibliothek \textit{free-theorems} bietet.

Die Datei sollte sich ganz normal über \texttt{ghc test.hs} kompilieren lassen. Das Programm fragt nach einer
Funktionssignatur der Form \texttt{funktionsname :: funktionstyp}, zu der dann das entsprechende freie Theorem
generiert wird, außerdem werden entsprechende \textit{lift}-Definitionen angezeigt.

Darüber hinaus bietet \texttt{test} einige Befehle, die insbesondere hilfreich sein können, um die Interna der
Bibliothek \textit{free-theorems} nachzuvollziehen.

\begin{table}[ht]
\begin{tabular}{ | l | l | }
\hline
Befehl & Funktion \\
\hline
:v & Schaltet den \textit{verbosen} Modus an bzw. aus.\\
& Dieser Modus zeigt zusätzlich zu den Ergebnisformeln die auftretenden \\
& Zwischendarstellungen an. \\
:basic & Setzt die Sprache auf \texttt{BasicSubset} \\
:fix & Setzt die Sprache auf \texttt{(SubsetWithFix EquationalTheorem)} \\
:fix! & Setzt die Sprache auf \texttt{(SubsetWithFix InquationalTheorem)} \\
:seq & Setzt die Sprache auf \texttt{(SubsetWithSeq EquationalTheorem)} \\
:seq! & Setzt die Sprache auf \texttt{(SubsetWithSeq InequationalTheorem)} \\
:q & Beendet das Programm \\
\hline
\end{tabular}
\caption{Befehle für das Testprogramm}
\label{tab:commands}
\end{table}

Die Datei \texttt{KnownDeclarations.hs}, die von \texttt{test.hs} eingebunden wird, ist eine modifizierte Version
derselben Datei, die sich im Original-Paket zum Paket namens \texttt{free-theorems-webui-0.2.1.1} finden lässt, wobei es sich um
die Weboberflächen-Implementierung für \textit{free-theorems} handelt. Der Ordner
\texttt{free-theorems-webui-0.2.1.1} enthält diese Implementierung, wobei die \texttt{KnownDeclarations.hs} dort
ebenfalls modifiziert wurde.
