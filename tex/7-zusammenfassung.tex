% % % % % % % % % % % % % % % % % % % % % % % % % % % % % % % % % % % % % % % % %
\section{Zusammenfassung und Ausblick}
% % % % % % % % % % % % % % % % % % % % % % % % % % % % % % % % % % % % % % % % %

In dieser Arbeit wurde eine bereits existierende Haskell-Bibliothek zur automatischen Generierung freier Theoreme
beschrieben und eine Erweiterung dieser Bibliothek zusammengefasst, mit der es möglich ist, freie Theoreme auch im Zusammenhang
mit Typkonstruktorklassen zu generieren.

In Kapitel \ref{sec:freie-theoreme} wurde vorgestellt, wie nach Wadler \cite{wadler} die Interpretation von Typen als Typrelationen
vonstatten geht und wie sich daraus wertvolle Theoreme ableiten lassen. Insbesondere wurde die Herleitung von Theoremen
in der Anwesenheit von Typkonstruktorklassen und Typkonstruktorvariablen erläutert, wie sie von Voigtländer \cite{voigtlander}
beschrieben wird.

Schließlich wurde die Bibliothek \textit{free-theorems} \cite{freetheorems} beschrieben und anhand einiger Beispieldaten
veranschaulicht. Die Technik zur Herleitung von Theoremen zu Typkonstruktorklassen wurde schließlich verwendet, um der
beschriebenen Bibliothek die entsprechende Funktionalität hinzuzufügen.

Dazu war es notwendig, die internen Datenstrukturen so zu erweitern, dass sie die hinzukommenden syntaktischen und formalen
Ausdrücke speichern können. Da die Bibliothek auf etwas älteren Modulen aufbaute, wurden hierbei auch
Kompatibilitätsprobleme beseitigt, die ein Kompilieren der Bibliothek mit neueren Versionen des verwendeten GHC verhinderte.
%Anhand dieser wurde dann in \ref{sec:free-theorems}
%die Bibliothek \textit{free-theorems} \cite{freetheorems} erläutert und anhand einiger Beispieldaten veranschaulicht.

%Da diese Bibliothek vorher nicht mit Typkonstruktorklassen umgehen konnte, lag der Fokus dieser Arbeit darauf, die Bibliothek dermaßen
%zu erweitern, dass sie auch mit Typkonstruktorklassen arbeiten kann. Nebenbei wurden einige Kompatibilitätsprobleme beseitigt,
%die ein Kompilieren der Bibliothek mit aktuellen Versionen des verwendeten GHC \cite{ghc} verhinderte

Und natürlich wurde an einigen Stellen die Funktionalität der Bibliothek verändert, um die Theoremgenerierung auf
Typkonstruktorvariablen auszuweiten. So wurden beispielsweise neue abzufangende Fehlerfälle eingeführt, die
Überführung der Syntax in entsprechende relationale Strukturen wurde erweitert und das \textit{Abrollen} der
Parametrizitätsaussage musste an die erweiterte Vorgehensweise angepasst werden.

Schließlich wurde kurz darauf eingegangen, inwiefern der Fixpunktoperator und Striktheitsoperator dafür sorgen,
dass freie Theorome in der Tat nicht ganz so simpel sind, wie sie im Großteil dieser Arbeit dargestellt werden. Kapitel
\ref{sec:striktheit-und-rekursion} hat diese Probleme erläutert und hat Lösungen von Wadler, Voigtländer und Johann angedeutet,
indem nämlich die Sprache Haskell in Teilsprachen aufgeteilt wird, wobei in jeder Teilsprache unterschiedlich starke Einschränkungen
an die konstruierten Relationen gestellt werden.

Durch die Erweiterung der Bibliothek auf Typkonstruktorklassen ist es nun also möglich, freie Theoreme für einen breiteren
Rahmen an Programmen zu generieren, da man auf dieses wichtige Sprachelement nicht länger verzichten muss.

\subsection{Ausblick}

In der Erweiterung der Bibliothek wird bisher nur die Sprache \texttt{BasicSubset} unterstützt, in der weder der Fixpunktoperator
noch Striktheit vorgesehen sind. Da es sich bei diesen Sprachelementen jedoch um praktisch unverzichtbare Konstrukte
in Haskell handelt, wäre es wünschenswert, die Implementierung dahingehend zu erweitern, dass auch
die Teilsprachen \texttt{SubsetWithFix} und \texttt{SubsetWithSeq} unterstützt werden.

%Verschiedene Quellen stellen  vor, freie Theoreme für Typsignaturen zu generieren, die Typkonstruktorklassen
%verwenden. In dieser Arbeit findet die Vorgehensweise von Voigtländer \cite{voigtlander} Anwendung, die die relationale Interpretation
%von Typsignaturen erweitert um eine Interpretation für Typkonstruktorvariablen-Abstraktion: Es wird über Funktionen
%quantifiziert, die ihrerseits Typrelationen auf neue Relationen abbilden.

%Mithilfe dieser Erweiterung lässt sich die Bibliothek jetzt auf einen breiteren Rahmen an Haskell-Programmen anwenden, da
%Typsignaturen, die Typkonstruktorklassen verwenden, nicht länger wegfallen.


%Es wurde bereits in Kapitel \ref{sec:striktheit-und-rekursion} darauf eingegangen, inwiefern Striktheit und der Fixpunktoperator
%eine Rolle bei der Gegnerierung freier Theoreme spielen. In der Erweiterung der Bibliothek wurden diese bisher außer Acht gelassen,
%sie funktioniert nur mit der \texttt{BasicLanguage} korrekt. Aber natürlich wäre es wünschenswert, auch die übrigen
%Teilsprachen zu unterstützen.

%Die umgesetzte Erweiterung basierte zum großen Teil auf einer Arbeit von Voigtländer \cite{voigtlander}, deren Betrach

In der umgesetzten Implementierung ist nur die Typsorte $* \rightarrow *$ vorgesehen.
Aktuelle Spracherweiterungen des
GHC Compilers ermöglichen hingegen sogar benutzerdefinierte Typsorten \cite{yorgey}. Auch
hier wäre noch Raum für Verbesserungen.

Die Theorie hinter der Erweiterung um Typkonstruktorklassen wendet dasselbe Vorgehen an wie
Voigtländer \cite{voigtlander}. Es existieren weitere Ansätze, um freie Theoreme zu höheren Sorten
zu generieren, so zum Beispiel von Aktey, der ein Modell aufstellt, in dem Sorten als reflexive
Graphen dargestellt werden \cite{atkey}.

Wirklich komfortabel wird der Umgang mit der Bibliothek \textit{free-theorems} erst, wenn ein angenehmes Benutzerinterface verfügbar ist.
Bei der Entstehung dieser Arbeit wurde beispielsweise ein kleines Tool geschrieben, das Zugriff auf die Funktionalitäten
der Bibliothek bietet. Es verrichtet seine Arbeit, bietet aber keinen echten Komfort beim Erzeugen der Theoreme. Zudem
läuft es lediglich in der Kommandozeile.

Angenehmer stellt sich die Verwendung dar, wenn beispielsweise ein Webinterface wie das Paket \textit{free-theorems-webui}
\cite{freetheoremswebui} verwendet wird. Nicht nur wird die Eingabe dadurch intuitiver und einfacher, es ist auch eine
übersichtlichere Darstellung des resultierenden Theorems möglich.

\textit{free-theorems-webui} importiert dabei eine Datei vordefinierter Deklarationen, sodass bekannte Typen aus der
Prelude verwendet werden können und man Funktionssignaturen zu bekannten Funktionen direkt über deren Namen
angeben kann. Bei der Umsetzung der vorliegenden Arbeit wurden Deklarationen für \texttt{Functor} und \texttt{Monad}
eingebaut. Doch nicht nur gibt es weitere wichtige Typklassen, die jetzt ebenfalls verwendet werden können, es gibt
auch viele Funktionen, die mit Typkonstruktorvariablen arbeiten und daher bisher nicht verwendbar waren.

Es bietet sich also an, die Liste der \texttt{KnownDeclarations} auszubauen und sich zu überlegen, welche
Typkonstruktorklassen und Funktionen sinnvoll sind.