% % % % % % % % % % % % % % % % % % % % % % % % % % % % % % % % % % % % % % % % %
\section{Einleitung}
% % % % % % % % % % % % % % % % % % % % % % % % % % % % % % % % % % % % % % % % %

Die funktionale Programmiersprache Haskell bietet ein komplexes Typsystem, das auf der Typinferenz nach Hindley-Milner basiert \todo{cite}.
Zu jedem Ausdruck lässt sich eine Typsignatur herleiten, die den allgemeinsten Typen dieses Ausdrucks darstellt. Außerdem
kann der Programmierer zu jedem Ausdruck eine Typsignatur angeben, was zur Übersicht beiträgt und Fehlern vorbeugt.
Das Typsystem ist so mächtig, dass die Typsignatur allein schon sehr viele Rückschlüsse über die zugehörige Funktion ermöglicht.
Betrachten wir beispielhaft die folgende Signatur.

\begin{minted}{haskell}
test :: (a -> b) -> [a] -> [b]
\end{minted}

Allein durch diesen Typen lässt sich bereits erahnen, was sich ungefähr hinter dieser Funktion verbirgt. Die Funktion nimmt
eine Funktion und eine Liste als Argumente. Die Funktion muss dabei den Typen der Listenelemente auf einen anderen Typen
abbilden. Als Ergebnis liefert die Funktion den Rückgabetypen der übergebenen Funktion. Es ist anzunehmen, dass die Funktion
\texttt{test} die Eingabeliste mithilfe der übergebenen Funktion manipuliert und die resultierende Liste zurückgibt.

Haskell liefert eine solche Funktion in der Prelude, man kennt sie unter dem Namen \texttt{map}. Dass sich \texttt{test} und
\texttt{map} eine Typsignatur teilen, lässt also Rückschlüsse auf eine ähnliche Funktionsweise zu. Doch diese Ähnlichkeit
ist nicht auf vage Aussagen beschränkt, man kann ganz systematisch korrekte Aussagen zu beliebigen Typsignaturen
herleiten, wie \cite{xyz} \todo{cite} zeigt.

Die Herleitung freier Theoreme ist so generisch, dass sie auch programmtisch erledigt werden kann. Genau das macht
\cite{bla} \todo{cite} - hierbei handelt es sich um die Bibliothek \textit{free-theorems}, die es ermöglicht, Haskell-Code einzulesen
und aus beliebigen Typsignaturen Formeln zu generieren, die die entsprechenden freien Theoreme repräsentieren. Zudem gibt es
mit \cite{x} \todo{cite} eine Webanwendung, die einen Zugriff auf die Funktionen der Bibliothek ermöglicht.
Leider bietet die Bibliothek \textit{free-theorems} nicht sämtliche Sprachfunktionen von Haskell an. 

Ein wichtiger Aspekt
bei der Programmierung in Haskell sind Typklassen. Sie ermöglichen es, Datenstrukturen in Klassen einzuordnen, und für diese
Klassen Funktionen anzugeben, die pro Datentyp implementiert werden können. Kurz gesagt erlauben Typklassen Ad-Hoc-Polymorphismus in Haskell.

Zwar unterstützt die Bibliothek Typklassen, allerdings nur für Datentypen, die keinen Typen als Parameter erwarten. Die
bekanntere \texttt{Monad}-Klasse beispielsweise ist für Datentypen definiert, die einen Typparameter haben. Eine Verwendung
mit der \texttt{free-theorems} Bibliothek ist somit nicht möglich, oftmals möchte man aber nicht auf solche Typklassen verzichten.
Grundsätzlich unterscheidet sich die Verwendung von Typkonstruktorklassen gar nicht gravierend von einfachen Typklassen.

Ich habe die Bibliothek um eine entsprechende Funktionalität erweitert. In der vorliegenden Arbeit wird erläutert, wie die Bibliothek
und insbesondere die Anpassungen für Typkonstruktorklassen aussehen. Dazu wird zunächst auf die Grundlagen eingegangen,
die benötigt werden, um die Theorie dahinter zu verstehen. Im folgenden Abschnitt geht es um das allgemeine Vorgehen beim
Herleiten freier Theoreme. Insbesondere wird dabei auch auf Typkonstruktorklassen eingegangen.

Schließlich wird die Bibliothek \textit{free-theorems} etwas genauer beschrieben, wobei insbesondere auf den Aufbau und die
grundsätzliche Aufteilung eingegangen wird, um dann im darauf folgenden Abschnitt zu erläutern, an welchen Stellen diese
Bibliothek erweitert werden muss, um die zusätzliche Funktionalität für Typkonstruktorklassen zu leisten.

In dieser Arbeit wird dabei größtenteils von einer vereinfachten Sprache ausgegangen, die weder Striktheit noch den Fixpunktoperator kennt.
Die ursprüngliche \texttt{free-theorems} Bibliothek berücksichtigt die Besonderheiten bereits, die sich durch Verwendung dieser
speziellen Operatoren ergeben. Die Erweiterungen für Typkonstruktorklassen ignorieren diese Besonderheiten bisher, hier
sind eventuell noch Anpassungen möglich. In Abschnitt \todo{ref} wird genauer auf diese Problematik eingegangen. \todo{?}