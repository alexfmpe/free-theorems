% % % % % % % % % % % % % % % % % % % % % % % % % % % % % % % % % % % % % % % % %
\section{Striktheit und Rekursion}
% % % % % % % % % % % % % % % % % % % % % % % % % % % % % % % % % % % % % % % % %

Bisher wurde eine Wahrheit größtenteils ignoriert: Die bisher generierten freien Theoreme sind nur unter ganz bestimmten
Bedingungen gültig. Beim Beispiel \texttt{$test :: [a] \rightarrow [a]$} wurde behauptet, dass die Funktion \texttt{test}
lediglich Ausdrücke aus der Eingabeliste in der Ausgabeliste weiderverwenden könne. Tatsächlich entspricht das nicht ganz der
Wahrheit, denn in Haskell ist zum Beispiel auch denkbar, dass $\bot$ (\textit{bottom}) zurückgegeben wird, beispielsweise
durch Nichtterminierung oder einen Laufzeitfehler.

free-theorems definiert verschiedene Untermengen der Programmiersprache Haskell, zu denen sich unterschiedlich komplexe
Theoreme generieren lassen. Ein Faktor, der nämlich Probleme bereiten kann, ist der Fixpunktoperator.
Die Bilbiothek definiert drei Untermengen:

\begin{itemize}
   \item BasicSubset
   \item SubsetWithFix
   \item SubsetWithSeq
\end{itemize}

Die erste Untermenge, BasicSubset, entspricht dem Girard-Reynoldschen Lambda-Kalkül \cite{bla}, erweitert um Datentypen und
Typklassen \cite{freetheorems}. Insbesondere sind aber keine undefinierten Werte erlaubt, die zum Beispiel durch Nichtterminierung,
Division durch Null oder Pattern Matching Fehler auftreten.
Das ändert sich in SubsetWithFix. Hier ist zusätzlich der Fixpunktoperator \textit{fix} verfügbar, und es müssen folglich undefinierte
Werte erlaubt werden. Was noch fehlt, ist die \textit{seq}-Funktion sowie Striktheits-Flags. Letztere können bei selbstdefinierten
Datentypen an die einzelnen Untertypen \todo{Richtige Bezeichnung für einzelne "Parameter" eines Datentyps?} annotiert werden,
um eine strikte Auswertung zu erzwingen.
Die Untermenge SubsetWithSeq ermöglicht schließlich die zuletzt genannten Aspekte: Es wird Striktheit durch \textit{seq}
eingeführt, dadurch sind dann auch Striktheitsannotationen in Datentypen möglich.

Es wird noch eine weitere Unterscheidung eingeführt: Zu SubsetWithFix und SubsetWithSeq kann noch unterschieden werden,
ob Gleichungs- oder Ungleichungsresultate gewünscht sind. \todo{equational and inequational results - bessere Übersetzung?}
Daraus ergeben sich insgesamt fünf Varianten, für die \cite{freetheorems} die folgende Schreibweise einführt, wobei $M$ die Menge
der verfügbaren \textit{Modelle} ist:

\begin{align}
M = \{(basic, =), (fix, =), (fix, \sqsubseteq), (seq, =), (seq, \sqsubseteq)\}
\end{align}