% !TEX TS-program = pdflatex
% !TEX encoding = UTF-8 Unicode

% This is a simple template for a LaTeX document using the "article" class.
% See "book", "report", "letter" for other types of document.

\documentclass[11pt]{article} % use larger type; default would be 10pt

\usepackage{courier}

\usepackage[utf8]{inputenc} % set input encoding (not needed with XeLaTeX)

\usepackage{listings}

\lstset{
  tabsize=2,
  frame=single,
  basicstyle=\footnotesize\ttfamily
}

%%% Examples of Article customizations
% These packages are optional, depending whether you want the features they provide.
% See the LaTeX Companion or other references for full information.

%%% PAGE DIMENSIONS
\usepackage{geometry} % to change the page dimensions
\geometry{a4paper} % or letterpaper (US) or a5paper or....
% \geometry{margin=2in} % for example, change the margins to 2 inches all round
% \geometry{landscape} % set up the page for landscape
%   read geometry.pdf for detailed page layout information

\usepackage{graphicx} % support the \includegraphics command and options

% \usepackage[parfill]{parskip} % Activate to begin paragraphs with an empty line rather than an indent

%%% PACKAGES
\usepackage{booktabs} % for much better looking tables
\usepackage{array} % for better arrays (eg matrices) in maths
\usepackage{paralist} % very flexible & customisable lists (eg. enumerate/itemize, etc.)
\usepackage{verbatim} % adds environment for commenting out blocks of text & for better verbatim
\usepackage{subfig} % make it possible to include more than one captioned figure/table in a single float
% These packages are all incorporated in the memoir class to one degree or another...

%%% HEADERS & FOOTERS
\usepackage{fancyhdr} % This should be set AFTER setting up the page geometry
\pagestyle{fancy} % options: empty , plain , fancy
\renewcommand{\headrulewidth}{0pt} % customise the layout...
\lhead{}\chead{}\rhead{}
\lfoot{}\cfoot{\thepage}\rfoot{}

%%% SECTION TITLE APPEARANCE
\usepackage{sectsty}
\allsectionsfont{\sffamily\mdseries\upshape} % (See the fntguide.pdf for font help)
% (This matches ConTeXt defaults)

%%% ToC (table of contents) APPEARANCE
\usepackage[nottoc,notlof,notlot]{tocbibind} % Put the bibliography in the ToC
\usepackage[titles,subfigure]{tocloft} % Alter the style of the Table of Contents
\renewcommand{\cftsecfont}{\rmfamily\mdseries\upshape}
\renewcommand{\cftsecpagefont}{\rmfamily\mdseries\upshape} % No bold!

%%% END Article customizations

%%% The "real" document content comes below...

\title{free-theorems: Syntaxbaum}
\author{Thomas Rossow}
%\date{} % Activate to display a given date or no date (if empty),
         % otherwise the current date is printed 

\begin{document}
\maketitle

\section{BasicSyntax}

Das Paket free-theorems nutzt intern die selbstdefinierte Datenstruktur BasicSyntax als vereinfachte Version eines Syntaxbaumes zu einem Haskellprogramm. Diese Datenstruktur beinhaltet lediglich die Konstrukte, die für die Generierung von freien Theoremen benötigt werden: Von Interesse sind hier hauptsächlich natürlich Funktionssignaturen, daneben aber auch mit \texttt{data} definierte Datentypen und - insbesondere für die Erweiterung - Klassendefinitionen und Klasseneinschränkungen. Intern nutzt das Paket die Parser aus Language.Haskell.Exts und transformiert den resultierenden Syntaxbaum in die BasicSyntax.

\subsection{Funktionssignaturen}

Das wichtigste Element sind Funktionssignaturen. Bei der Überführung in die BasicSyntax werden sämtliche Funktionsimplementationen ignoriert und lediglich die Signaturen übernommen.

\lstinputlisting[language=Haskell]{ast.hs}

\subsection{Eigene Datentypen}

Da auch Datentypen in Funktionssignaturen vorkommen können, müssen die Datentypdeklarationen ebenfalls vorliegen. Zu beachten ist hier, dass die Typen der jeweiligen Datenkonstruktoren in \texttt{BangTypeExpression} verpackt sind: Der Konstruktor \texttt{Banged} wird verwendet, wenn der entsprechende Typ als "strikt" deklariert ist, ansonsten wird \texttt{Unbanged} benutzt.

\lstinputlisting[language=Haskell]{ast2.hs}

\subsection{Eigene Typklassen}

Besonders wichtig für die Erweiteurng von free-theorems sind die Typklassen. Auch diese werden in die Deklarationsliste mit aufgenommen, wobei auch hier die tatsächlichen Implementierungen der Funktionen keine Rolle spielen, lediglich die Signaturen werden verwendet.

\lstinputlisting{ast3.hs}

Zu sehen ist zum einen die Deklaration einer Klasse sowie einer Funktion \texttt{test}, deren Typvariable \texttt{m} auf die Klasse Monad eingeschränkt ist. \texttt{classFuns} enthält eine Liste von Deklarationen in dieser Klasse. Für dieses Beispiel ergibt das die folgende Datenstruktur:

\lstinputlisting{ast4.hs}

\texttt{TypVarApp} wurde hierbei nachträglich in free-theorems eingebaut, da es ursprünglich nicht vorgesehen war, Typkonstruktorvariablen zu benutzen.

\end{document}
